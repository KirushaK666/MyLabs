\documentclass{article}
\usepackage{amsmath}
\usepackage{amsfonts}
\usepackage[russian]{babel}

\begin{document}

\noindent
и начальные условия в виде рядов Фурье:

\begin{equation}
    \left.
    \begin{aligned}
        f(x, t) &= \sum_{n=1}^{\infty} f_n(t) \sin{\frac{n\pi}{l} x}, & f_n(t) &= \frac{2}{l} \int\limits_{0}^{l} f(\xi, t) \sin{\frac{n\pi}{l} \xi} d\xi; \\
        \varphi(x) &= \sum_{n=1}^{\infty} \Phi_n \sin{\frac{n\pi}{l} x}, & \Phi_n &= \frac{2}{l} \int\limits_{0}^{l} \varphi(\xi) \sin{\frac{n\pi}{l} \xi} d\xi; \\
        \psi(x) &= \sum_{n=1}^{\infty} \Psi_n \sin{\frac{n\pi}{l} x}, & \Psi_n &= \frac{2}{l} \int\limits_{0}^{l} \psi(\xi) \sin{\frac{n\pi}{l} \xi} d\xi.
    \end{aligned}
    \right\} \tag{49}
\end{equation}

\noindent
Подставляя предполагаемую форму решения (48) в исходное уравнение (45)

\begin{equation*}
    \sum_{n=1}^{\infty} \sin{\frac{n\pi}{l} x} \left\{ -a^2 \left(\frac{n\pi}{l}\right)^2 u_n(t) - \ddot{u}_n(t) + f_n(t) \right\} = 0,
\end{equation*}

\noindent
видим, что оно будет удовлетворено, если все коэффициенты разложения равны нулю, т. е.

\begin{equation}
    \ddot{u}_n(t) + \left(\frac{n\pi}{l}\right)^2 a^2 u_n(t) = f_n(t). \tag{50}
\end{equation}

\noindent
Для определения $u_n(t)$ мы получили обыкновенное дифференциальное уравнение с постоянными коэффициентами. Начальные условия дают:

\begin{equation*}
    u(x, 0) = \varphi(x) = \sum_{n=1}^{\infty} u_n(0) \sin{\frac{n\pi}{l} x} = \sum_{n=1}^{\infty} \Phi_n \sin{\frac{n\pi}{l} x},
\end{equation*}

\begin{equation*}
    u_t(x, 0) = \psi(x) = \sum_{n=1}^{\infty} \dot{u}_n(0) \sin{\frac{n\pi}{l} x} = \sum_{n=1}^{\infty} \Psi_n \sin{\frac{n\pi}{l} x},
\end{equation*}

\noindent
откуда следует:

\begin{equation}
    \begin{cases}
        u_n(0) = \Phi_n, \\
        \dot{u}_n(0) = \Psi_n.
    \end{cases} \tag{51}
\end{equation}

\noindent
Эти дополнительные условия полностью определяют решение уравнения (50). Функцию $u_n(t)$ можно представить в виде

\begin{equation*}
    u_n(t) = u_n^{(1)}(t) + u_n^{(II)}(t),
\end{equation*}

\noindent
где

\begin{equation}
    u_n^{(1)}(t) = \frac{l}{n\pi a} \int\limits_{0}^{t} \sin{\frac{n\pi}{l} a (t - \tau)} \cdot f_n(\tau) d\tau \tag{52}
\end{equation}

\noindent
4 А. Н. Тихонов, А. А. Самарский


\noindent
\\

\noindent
есть решение неоднородного уравнения с нулевыми начальными условиями$^1)$ и

\begin{equation}
    u^{(1)}(t) = \Phi_n \cos{\frac{n\pi}{l} at} + \frac{l}{n\pi a} \Psi_n \sin{\frac{n\pi}{l} at} \tag{53}
\end{equation}

\noindent
— решение однородного уравнения с заданными начальными условиями. Таким образом, искомое решение запишется в виде

\begin{equation*}
    u(x, t) = \sum_{n=1}^{\infty} \frac{l}{n\pi a} \int\limits_{0}^{t} \sin{\frac{n\pi}{l} a (t - \tau)} \sin{\frac{n\pi}{l} x} \cdot f_n(\tau) d\tau +
\end{equation*}
\begin{equation}
    + \sum_{n=1}^{\infty} \left( \Phi_n \cos{\frac{n\pi}{l} at} + \frac{l}{n\pi a} \Psi_n \sin{\frac{n\pi}{l} at} \right) \sin{\frac{n\pi}{l} x}. \tag{54}
\end{equation}

\noindent
Вторая сумма представляет решение задачи о свободных колебаниях струны при заданных начальных условиях и была нами исследована ранее достаточно подробно. Обратимся к изучению первой суммы, представляющей вынужденные колебания струны под действием внешней силы при нулевых начальных условиях. Пользуясь выражением (49) для $f_n(t)$, находим:

\begin{equation*}
    u^{(1)}(x, t) = \int\limits_{0}^{t}\int\limits_{0}^{t} \left\{ \frac{2}{l} \sum_{n=1}^{\infty} \frac{l}{n\pi a} \sin{\frac{n\pi}{l} a (t - \tau)} \sin{\frac{n\pi}{l} x} \sin{\frac{n\pi}{l} \xi} \right\} f(\xi, \tau) d\xi d\tau =
\end{equation*}

\begin{equation}
    \begin{aligned}
        &= \int\limits_{0}^{t} \int\limits_{0}^{l} G(x, \xi, t - \tau) f(\xi, \tau) d\xi d\tau,  \\
        \end{aligned} \tag{55}
\end{equation}
\text{где} \\
\begin{equation}
    \begin{aligned}
        &G(x, \xi, t - \tau) = \frac{2}{l} \sum_{n=1}^{\infty} \frac{l}{n\pi a} \sin{\frac{n\pi}{l} a (t - \tau)} \sin{\frac{n\pi}{l} x} \sin{\frac{n\pi}{l} \xi}. 
    \end{aligned} \tag{56}
\end{equation}

\noindent
Выясним физический смысл полученного решения. Пусть функция $f(\xi, \tau)$ отлична от нуля в достаточно малой окрестности точки $M_0(\xi_0, \tau_0)$:

\begin{equation*}
    \xi_0 \le \xi \le \xi_0 + \Delta\xi, \quad \tau_0 \le \tau \le \tau_0 + \Delta\tau.
\end{equation*}

\noindent
Функция $\rho f(\xi, \tau)$ представляет плотность действующей силы; сила, приложенная к участку $(\xi_0, \xi_0 + \Delta\xi)$, равна

\begin{equation*}
    F(\tau) = \rho \int\limits_{\xi_0}^{\xi_0 + \Delta\xi} f(\xi, \tau) d\xi.
\end{equation*}

\noindent
\footnote{1) В этом можно убедиться непосредственно. Формула (52) может быть получена методом вариации постоянных. См. также мелкий шрифт в конце настоящего пункта.}

\end{document}
